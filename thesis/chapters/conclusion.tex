Community detection is a difficult field with no one-size-fits-all method. There is no unique, rigorous definition of what a community is; different algorithms optimise different objectives (e.g.\ modularity vs.\ flow compression), and results depend on the algorithm, the network, and the scale of structure of interest \citep{Fortunato_2016}. That is why we applied several methods, compared partitions with NMI and VI, and used consensus clustering to obtain a stable partition for interpretation rather than relying on a single run. It also motivated our choice to \emph{interpret} the detected modules with taxonomy and environment rather than to validate them by agreement with metadata. Below we summarise the findings, methodological lessons, and limitations.


\subsection{Summary of Findings}


We analysed microbial co-occurrence networks from the GRUMP database for the Atlantic and Pacific basins, using the same set of 3175 ASVs (nodes) and approximately 25\,000 edges per basin. Both networks are sparse (density $\approx 0.005$) and show similar mean degree and edge weights. The Atlantic has higher clustering (0.097 vs.\ 0.062), two connected components (with 99.6\% of nodes in the largest), and a larger diameter than the Pacific. We ran five community detection algorithms (Louvain, Infomap, Fast Greedy, Leading Eigenvector, Walktrap) and found that the number of communities varied strongly across methods (from 5 to 57), and that the Atlantic attained higher modularity $Q$ than the Pacific for every algorithm. Pairwise NMI between partitions was moderate (roughly 0.51--0.67), with greater agreement among modularity-based methods. We used iterative consensus clustering (Louvain, $n_P = 30$, $\tau = 0.5$) to obtain a stable partition for downstream analysis: the procedure converged in two iterations in both basins, yielding 11 consensus communities in the Atlantic ($Q = 0.61$) and 10 in the Pacific ($Q = 0.59$). Mean community size was 289 (Atlantic) and 318 (Pacific), with median 176 and 378 respectively---Pacific modules are larger on average and more even in size, while the Atlantic has one more community and at least one relatively small module. We interpreted these consensus modules using taxonomy (Phylum, Class) and node-level environmental summaries (temperature, salinity, depth, oxygen) and visualised the networks with a shared layout and multiple colourings. We did not validate partitions by alignment with metadata, but used metadata to attach ecological meaning to the structure.


\subsection{Methodological Insights}


Applying multiple algorithms made clear that there is no unique ``best'' partition when ground truth is unknown: modularity-based methods returned fewer, larger communities and higher $Q$, while Infomap and Walktrap often returned many more, smaller communities. Reporting NMI and VI between partitions quantified this variation and showed that modularity-based methods agree more with each other than with Infomap or Leading Eigenvector. Consensus clustering over many Louvain runs (following \citet{Lancichinetti_2012}) provided a single, reproducible partition for interpretation and avoided dependence on a single random run. We deliberately refrained from judging success by agreement with taxonomy or environment. Instead we used the framework of \citet{Fortunato_2016} to \emph{interpret} the structural modules with metadata rather than to validate them. The Methods section reviewed the role of significance testing (e.g.\ against the configuration model) and the resolution and detectability limits. In the analysis we reported modularity and partition similarity and used consensus as a stability device. We did not perform formal significance testing and leave null-model comparison and benchmark evaluation for future work (see Future Directions).


\subsection{Ecological Implications}


The consensus communities were interpreted in light of taxonomic composition (e.g.\ top phyla and classes per module) and environmental covariates (mean temperature, salinity, depth, oxygen per module). The seven-panel network figures (same layout, colourings by consensus module, temperature, latitude, depth, salinity, Phylum, and Class) allow a qualitative view of how structural clusters align with environmental or taxonomic gradients in each basin. We did not assert that modules ``match'' taxonomy or environment. Structural communities need not align with external covariates, and the goal was to attach ecological meaning (e.g.\ potential warm-surface vs.\ deep assemblages, or phylum-dominated vs.\ mixed modules) rather than to score the partition. Basin-specific patterns---higher modularity and one more community in the Atlantic, larger and more even-sized modules in the Pacific---suggest that the two basins differ in how clearly modular the co-occurrence structure is, consistent with the single-run algorithm comparison. These patterns can inform hypotheses about basin-level differences in microbial biogeography and the role of environmental gradients in shaping association structure.


\subsection{Limitations}


Several limitations affect the interpretation of our results. (1) \textbf{Resolution limit} \citep{Fortunato_2007}: modularity-based methods, including Louvain, cannot detect communities smaller than a scale of order $\sqrt{2m}$. Biologically meaningful small clusters may have been merged. (2) \textbf{Detectability}: in sparse networks, community structure can be information-theoretically undetectable \citep{Decelle_2011}. We did not test whether the GRUMP networks lie above or below the detectability threshold. (3) \textbf{Sparsity}: both networks are very sparse ($\approx 25$k edges for 3175 nodes), so modularity and partition quality should be interpreted cautiously. (4) \textbf{Data and inference}: the node set and edges are fixed by the GRUMP filtering and SPIEC-EASI (NetCoMi) pipeline. Different filters or inference methods would yield different networks and hence different community structure. (5) \textbf{Single consensus strategy}: we used Louvain-based iterative consensus only. Consensus over multiple algorithms or other stability schemes could yield different partitions. (6) We did not perform formal significance testing of observed modularity against the configuration model in the reported pipeline.


\subsection{Future Directions}


Natural extensions include applying the same pipeline to other ocean basins (e.g.\ Indian Ocean, Southern Ocean) and to temporal or seasonal slices of the data to compare community structure across space and time. Deeper integration with environmental drivers (e.g.\ gradients, fronts, nutrient availability) could help explain why the Atlantic shows higher modularity and a different distribution of community sizes than the Pacific. On the methodological side, \emph{formal significance testing} of observed modularity (or other structure) against the configuration model would assess whether the detected communities are statistically meaningful rather than due to chance; this was not carried out in the present analysis. Comparing algorithms on benchmark networks (e.g.\ LFR) with planted partitions would characterise their behaviour before application to real data. Exploring consensus over multiple algorithms (e.g.\ Louvain, Infomap, Walktrap) or alternative stability criteria could yield more robust or interpretable partitions for ecological interpretation.
